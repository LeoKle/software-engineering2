\subsection*{Aufgabe 2: Testfallanalyse mittels Äquivalenzklassen und Grenzwertanalyse}

\subsubsection*{EmailChecker}
(a) Äquivalenzklassen\\
\begin{minipage}[t]{0.48\textwidth}
\textbf{Gültige Klassen:}
\begin{itemize}
    \item Simple: \\ name@adresse.de
    \item Mit Subdomain: \\ name1.name2@example.subdomain.org
    \item Local-Part: \\ name1.name2+tag@example.org
    \item Sonderzeichen (! \_ - etc.): \\ name!@domain.com
\end{itemize}
\end{minipage}
\hfill
\begin{minipage}[t]{0.48\textwidth}
\textbf{Ungültige Klassen:}
\begin{itemize}
    \item kein @-Zeichen: \\ name1.name2.de
    \item mehrere @-Zeichen: \\ name1@name2@de
    \item keine Domain: \\ name1@name2
    \item Leerzeichen: \\ name1 @name2.de
\end{itemize}
\end{minipage}

\vspace{1em}

(b) Grenzwertanalyse
\begin{itemize}
    \item Null
    \item Empty String
    \item Whitespace String
\end{itemize}

(c) Testfälle\\
\begin{tabular}{ c | c }
 Eingabe & Erwartetes Ergebnis \\
 \hline
 ``name@domain.de'' & True  \\
 ``name.name@domain.de'' & True  \\
 ``name-name@domain.com'' & True  \\
 ``name!name@domain.com'' & True  \\
 \hline
 Null & False  \\  
 Empty String & False  \\
 Whitespace String & False  \\
 ``name.@@domain.de'' & False \\
 ``@domain.de'' & False \\
 ``name@'' & False \\
 ``name@domain'' & False \\
 ``name@domain.'' & False \\
 ``name @domain.de'' & False \\
\end{tabular}

\subsubsection*{PasswordChecker}
(a) Äquivalenzklassen\\
\begin{minipage}[t]{0.48\textwidth}
\textbf{Gültige Klassen:}
\begin{itemize}
    \item minimale Länge: \\ Abcdef1!
    \item maximale Länge: \\ Abcdefghijklmnopqr1!
    \item beliebige Reihenfolge: \\ !Bcdefghijklmnopqr1!
\end{itemize}
\end{minipage}
\hfill
\begin{minipage}[t]{0.48\textwidth}
\textbf{Ungültige Klassen:}
\begin{itemize}
    \item Null
    \item minimale Länge: \\ Abcd1! 
    \item maximale Länge: \\ Abcdefghijklmnopqrstuvw1! 
    % \item kein Uppercase: \\ abcdefghijklmnopqr1! 
    % \item kein Lowercase: \\ ABCDEFGHIJKLMNOPQR1! 
    % \item keine Zahl: \\ AbcdefghijklmnopqrS! 
    % \item kein Sonderzeichen: \\ Abcdefghijklmnopqr1a
    % nicht erlaubtes Sonderzeichen
    % nur Zahlen, Lowercase, Uppercase, Sonderzeichen
    % mehrere von Zahlen, Lowercase, Uppercase, Sonderzeichen fehlen
\end{itemize}
\end{minipage}

\vspace{1em}

(b) Grenzwertanalyse
\begin{itemize}
    \item Null
    \item Passwort mit Whitespace
    \item nicht erlaubte Sonderzeichen
    \item Escape-Characters 
\end{itemize}

(c) Testfälle\\
\begin{tabular}{ c | c }
 Eingabe & Erwartetes Ergebnis \\
 \hline
 Abcdef1! & True  \\
 Abcdefghijklmnopqr1!! & True  \\
 !Bcdefghijklmnopqr1! & True  \\
 \hline
 Null & False \\  
 Abcd1!  & False \\  
 Abcdefghijklmnopqrstuvw1!  & False \\  
\end{tabular}

\subsubsection*{RegistrationValidator}
(a) Äquivalenzklassen\\
\begin{minipage}[t]{0.48\textwidth}
\textbf{Gültige Klassen:}
\begin{itemize}
    \item Gültige Email, Password und Username: \\ name@domain.de, Abcdef1!, validUser123
\end{itemize}
\end{minipage}
\hfill
\begin{minipage}[t]{0.48\textwidth}
\textbf{Ungültige Klassen:}
\begin{itemize}
    \item Ungültige Email: \\ name1.name2.de
    \item Ungültige Password: \\ Abcdef1!
    \item Ungültige Username: \\ user name
\end{itemize}
\end{minipage}

\vspace{1em}

(b) Grenzwertanalyse
\begin{itemize}
    \item Null
    \item Mehrere Leerzeichen 
\end{itemize}

(c) Testfälle\\
\begin{tabular}{ c | c }
 Eingabe & Erwartetes Ergebnis \\
 \hline
 name@domain.de, Abcdef1!, validUser123 & True  \\
 \hline
 Null & False  \\  
\end{tabular}

\subsubsection*{UsernameChecker}
(a) Äquivalenzklassen\\
\begin{minipage}[t]{0.48\textwidth}
\textbf{Gültige Klassen:}
\begin{itemize}
    \item Alphabeten und Zahlen: \\ validUser123
    \item Undercore: \\ user\_name
    \item Nur Zahlen: \\ 123456789
    \item Nur Alphabeten: \\ abcdefg
\end{itemize}
\end{minipage}
\hfill
\begin{minipage}[t]{0.48\textwidth}
\textbf{Ungültige Klassen:}
\begin{itemize}
    \item Null
    \item Name mit Leerzeichen: \\ user name
    \item Special Chars: \\ user@name
    \item minimale Länge: \\ abc
    \item maximale Länge: \\ thisusernameiswaytoolongtobevalid
\end{itemize}
\end{minipage}

\vspace{1em}

(b) Grenzwertanalyse
\begin{itemize}
    \item Null
    \item Username mit mehrere Underscores
    \item Username mit Lerrzeichen
    \item Sonderzeichen 
\end{itemize}

(c) Testfälle \\
\begin{tabular}{ c | c }
 Eingabe & Erwartetes Ergebnis \\
 \hline
 validUser123 & True  \\
 user\_name & True \\
 123456789 & True \\
 abcdefg & True \\
 \hline
 Null & False  \\  
 user name & False \\
 user@name & False \\
\end{tabular}