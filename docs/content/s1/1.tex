\subsection*{Aufgabe 1: Ableitung von User Stories}

\subsubsection*{EmailChecker}
Als \textbf{[Nutzer]} möchte ich, \textbf{[dass meine E-Mail auf Richtigkeit geprüft wird]}, damit ich \textbf{[keine falsche angebe / kontaktiert werden kann]}.

Akzeptanzkriterien:
\begin{itemize}
    \item Leerer String wird abgelehnt
    \item E-Mail beinhaltet genau ein \gq{@}-Zeichen
    \item E-Mail beinhaltet gültigen Teil vor dem \gq{@}-Zeichen:
    \\kein leerer String
    % \\1-63 Zeichen
    % \\A-Z,a-z,0-9
    \item E-Mail beinhaltet gültigen Teil nach dem \gq{@}-Zeichen:
    \\kein leerer String
    % \\1-63 Zeichen
    % \\A-Z,a-z,0-9
    \\beinhaltet Punkt
    \\endet nicht mit einem Punkt
    \item beinhaltet keine Leerzeichen
    \item E-Mail wird akzeptiert (gilt als gültig), wenn alle Kriterien erfüllt sind
\end{itemize}

\subsubsection*{PasswordChecker}
Als \textbf{[Team]} möchten wir \textbf{[Anforderungen an Passwörter festlegen]}, damit wir \textbf{[unseren Nutzer eine gewisse Sicherheit bieten können]}.

Akzeptanzkriterien:
\begin{itemize}
    \item Leere Passwörter sind nicht erlaubt TODO: sind sie anscheinend doch?
    \item Null wird abgelehnt
    \item Passwörter beinhalten min. 8 Zeichen
    \item Passwörter beinhalten max. 20 Zeichen
    \item Passwörter muss ein Großbuchstabe beinhalten
    \item Passwörter muss ein Kleinbuchstabe beinhalten
    \item Passwörter muss eine Zahl beinhalten
    \item Passwörter muss ein Sonderzeichen beinhalten
    \item Passwort wird akzeptiert, wenn es alle Kriterien erfüllt
\end{itemize}

\subsubsection*{RegistrationValidator}
Als \textbf{[System]} möchte ich \textbf{[Registrierungsdaten validieren können]}, damit ich \textbf{[nur korrekte Nutzerdaten gespeichert werden].}

Akzeptanzkriterien:
\begin{itemize}
    \item Nutzt EmailChecker \\ und erfüllt dessen Kriterien
    \item Nutzt PasswordChecker \\ und erfüllt dessen Kriterien
    \item Nutzt UsernameChecker \\ und erfüllt dessen Kriterien
    \item Nur erfolgreich wenn alle drei Komponenten (Username, Email, Password) gültig sind
    \item Implementierungen (Subklassen) sind nicht zugänglich außer über eigene Public Methoden, welche die Methoden dieser aufrufen
\end{itemize}

\subsubsection*{UsernameChecker}
Als \textbf{[registrierter Nutzer]} möchte ich \textbf{[einen gültigen Benutzernamen wählen können]}, damit ich \textbf{[mich eindeutig identifizieren kann].}

Akzeptanzkriterien:
\begin{itemize}
    \item Null wird abgelehnt
    \item Username muss min. 4 Zeichen beinhalten
    \item Username darf max. 15 Zeichen beinhalten
    \item erlaubte Zeichen A-Z,a-z,0-9, Unterstrich \_
    \item keine Leerzeichen oder Sonderzeichen (außer \gq{\_})
    \item Username wird akzeptiert (gilt als gültig), wenn alle Kriterien erfüllt sind    
\end{itemize}