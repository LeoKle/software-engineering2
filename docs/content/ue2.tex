\section*{Aufgabe 1: Richtig oder Falsch?}

\begin{enumerate}
\item Das Agile Manifest schätzt \gq{Prozesse und Werkzeuge} höher als
\gq{Individuen und Interaktionen}:
\newline \textbf{Falsch} $\rightarrow$ \gq{Individuals and interactions over processes and tools}

\item Im Wasserfallmodell ist der Kunde hauptsächlich in den Phasen
der Anforderungsanalyse und beim Testen involviert.
\newline \textbf{Richtig} $\rightarrow$ Der Kunde ist in der Anforderungsanalyse involviert, um Spezifikationen / ein Lastenheft festzulegen. Später beim Testen wird das Produkt vom Kunden wieder geprüft, ob es seinen Erwartungen / dem Lastenheft entspricht.

\item Kanban vermeidet feste Zeitrahmen für die Fertigstellung von
Arbeitspaketen und fokussiert stattdessen auf kontinuierlichen
Durchfluss und Pull-Prinzipien.
\newline \textbf{Richtig} $\rightarrow$ Kanban setzt nicht wie Scrum auf Sprints, also feste Zeitrahmen. Stattdessen wird auf einen kontinuierlichen Arbeitsfluss und das Pull-Prinzip gesetzt, sofern Kapazitäten frei sind.

\item Das V-Modell erlaubt eine flexible Anpassung der Projektphasen
und -ergebnisse während des Entwicklungsprozesses.
\newline \textbf{Falsch} $\rightarrow$ Das V-Modell ist weiterhin ein klassisches Modell. Es erweitert das Wasserfall-Modell um den Punkt der Qualitätssicherung und leitet Testfälle aus den Phasen der Analyse, des Entwurfs und der Implementierung ab.

\item Agile Modelle wie Scrum erfordern, dass der Kunde nur zu
Beginn und am Ende des Projekts involviert ist.
\newline \textbf{Falsch} $\rightarrow$ Scrum bindet den Kunden kontinuierlich ein, z.B. indem dieser den Product-Owner stellt oder durch Teilnahme an Reviews nach Sprints.

\item Kanban erfordert, dass Arbeitspakete immer in der Reihenfolge
abgeschlossen werden, in der sie begonnen wurden.
\newline \textbf{Falsch} $\rightarrow$ Kanban limitiert nur die maximale Anzahl an Arbeitspaketen, die zu einem Zeitpunkt in Bearbeitung sein dürfen. Eine feste Reihenfolge wird nicht festlegt.

\end{enumerate}

\section*{Aufgabe 2: Agile vs. Klassische Vorgehensmodelle}

\begin{enumerate}
\item Startup: \textbf{agiles Vorgehensmodell} $\rightarrow$ Die im Text dargestellten Anforderungen sind häufige Updates und schnelle Anpassungen an Nutzerfeedback, welche für die agilen Methoden sprechen. Da es sich um ein Startup handelt, würden agile Methoden außerdem den Vorteil bieten, dass man sich schnell an Personaländerungen und/oder Kapitalveränderungen anpasse könnte.

\item Bank: \textbf{klassisches Vorgehensmodell} $\rightarrow$ Es gibt einen festen Zeitplan / Zeitlimit sowie ein festes Budget, diese erfordern Planung zur Einhaltung. Die Anwendung unterliegt hohen Sicherheits-Standards, dies erfordert sorgfältige Prüfung.   

\item Smart-Home: \textbf{agiles Vorgehensmodell} $\rightarrow$ Es wird Flexibilität auf einen sich schnell ändernden Markt gefordert. Agile Methoden erlauben inkrementelle Entwicklung, neue Funktionen / neue Gerätetypen können schrittweise integriert werden. Dies erfordert vermehrte Deployments.
\end{enumerate}

\section*{Aufgabe 3: Projektdurchführung nach Wasserfallmodell und Scrum}

\subsection*{3.1:}
\subsubsection*{Anforderungsdefinition:}
\begin{itemize}
\item Klärung mit dem Hausbesitzer, welche Räume / Bereiche renoviert werden sollen
\item Festlegung eines Budgets und Zeitplans
\item Prüfung rechtlicher Vorgaben: Werden Baugenehmigungen benötigt? Welche Vorschriften müssen eingehalten werden?
\item \textbf{Abschlusskriterien:} Dokumentation der Änderungen \& Abnahme dieser durch Besitzer, Architekten / Bauunternehmen
\end{itemize}

\subsubsection*{Entwurf:}
\begin{itemize}
\item Erstellung von Architektenplänen
\item Erstellung von Materiallisten
\item Auswahl von Bauunternehmen
\item \textbf{Abschlusskriterien:} Freigabe alle Pläne und Bauunterlagen
\end{itemize}

\subsubsection*{Entwicklung:}
Umsetzung der Arbeiten basierend auf den Plänen
\begin{itemize}
\item Abbrucharbeiten abgeschlossen
\item neue Wände, Böden, Kabel, Rohre, Dämmung, Fenster eingebaut
\item Abschlusskriterien: Bauabnahme durch Architekten/Bauleitung
\end{itemize}

\subsubsection*{Installation \& Test:}
Einbau und Testen aller Systeme z.B. Heizung, Elektrik, Wasser
\begin{itemize}
\item Mängelprotokoll erstellt und behoben
\item Abschlusskriterien: Haus ist bewohnbar
\end{itemize}

\subsubsection*{Wartung:}
\begin{itemize}
\item Wartung von Heizung, Elektrik und Wasser
\item ggf. Garantien beanspruchen o. Reparaturen beauftragen
\end{itemize}

\subsection*{3.2:}

\subsubsection*{Sprint 1:}

\textbf{Sprint-Ziel:} Erstellung der Planungsgrundlage für die Renovierung\\
\textbf{Ausgewählte Items:}
\begin{itemize}
\item Besprechung mit dem Hausbesitzer über gewünschte Änderungen (z.B. Raumaufteilung, energetische Sanierung).
\item Architektenpläne erstellen $\rightarrow$ Bestandsaufnahme des Gebäudes und erste Entwürfe für Grundrisse und Innenraumgestaltung erstellen.
\item Materialplanung $\rightarrow$ Auswahl von Materialien und deren Zulieferer eingrenzen
\end{itemize}

\textbf{Daily Scrum:}
\begin{itemize}
\item Team bespricht Fortschritte bei der Erstellung der Pläne und der Analyse der Anforderungen.
\item Hindernisse werden identifiziert (z. B. fehlende Genehmigungen oder unklare Anforderungen).
\end{itemize}

\textbf{Sprint Review:}
\begin{itemize}
\item Präsentation der ersten Entwürfe der Architektenpläne.
\item Feedback vom Hausbesitzer einholen und Anpassungen dokumentieren.
\end{itemize}


\subsubsection*{Sprint 2:}

\textbf{Sprint-Ziel:} Abschluss der Planungsphase / Vorbereitung auf die Bauarbeiten\\
\textbf{Ausgewählte Items:}
\begin{itemize}
\item Finalisierung der Architektenpläne - Anpassung basierend auf Feedback aus Sprint 1
\item Genehmigungen einholen
\item Bauunternehmen, Materialien und Zulieferer festlegen
\end{itemize}

\textbf{Daily Scrum:}
\begin{itemize}
\item Fortschrittskontrolle zur Planungsanpassung und Genehmigungsabwicklung.
\item mögliche Verzögerungen (z. B. langsame behördliche Prozesse) besprechen
\end{itemize}

\textbf{Sprint Review:}
\begin{itemize}
\item Präsentation der finalen Architektenpläne für die Bauarbeiten.
\item Verbindliche Beauftragung der Firmen.
\end{itemize}